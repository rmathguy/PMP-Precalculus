%%% Preamle% '{~{~{
%%% This is a template for making up an AMS-LaTeX file


%%% The following command chooses the default 10 point type.
%%% To choose 12 point, change it to
%%% \documentclass[12pt]{amsart}
\documentclass{amsart}

%%% The following command loads the amsrefs package, which will be
%%% used to create the bibliography:
\usepackage[lite]{amsrefs}

%%% The following command defines the standard names for all of the
%%% special symbols in the AMSfonts package, listed in
%%% http://www.ctan.org/tex-archive/info/symbols/math/symbols.pdf
\usepackage{amssymb}



%%% The following commands allow you to use \Xy-pic to draw
%%% commutative diagrams.  (You can omit the second line if you want
%%% the default style of the nodes to be \textstyle.)
\usepackage[all,cmtip]{xy}
\let\objectstyle=\displaystyle

%%% If you'll be importing any graphics, uncomment the following
%%% line.  (Note: The spelling is correct; the package graphicx.sty is
%%% the updated version of the older graphics.sty.)
\usepackage{tikz}
\usepackage{graphicx}
\usepackage{pgfplots}
\pgfplotsset{compat=1.18, width=10cm}
\graphicspath{ {./images/} }

\usepackage[parfill]{parskip}

\usepackage{tcolorbox}

\usepackage{dirtytalk}

\usepackage{multirow}

\usepackage{float}

\usepackage{mathptmx}
\usepackage[11pt]{moresize}

\usepackage{etoolbox}
\apptocmd{\lim}{\limits}{}{}

\tcbuselibrary{theorems}

\newtcbtheorem[number within=section]{thm}{Theorem}%
{colback=white!5,colframe=green!35!blue,fonttitle=\bfseries}{th}

\usepackage{kantlipsum} % for text filler

\usepackage[belowskip=-1pt,aboveskip=0pt]{caption}

\setlength{\intextsep}{10pt plus 2pt minus 2pt}

\setlength{\textwidth}{\paperwidth}

\addtolength{\textwidth}{-2in}
\calclayout



%%% This part of the file (after the \documentclass command,
%%% but before the \begin{document}) is called the ``preamble''.
%%% This is where we put our macro definitions.

%%% Comment out (or delete) any of these that you don't want to use.


\newcommand{\R}{\mathbb{R}}
\newcommand{\C}{\mathbb{C}}
\newcommand{\Z}{\mathbb{Z}}

%%%-------------------------------------------------------------------
%%%-------------------------------------------------------------------
%%% The Theorem environments:% '{~{~{
%%%
%%%
%%% The following commands set it up so that:
%%% 
%%% All Theorems, Corollaries, Lemmas, Propositions, Definitions,
%%% Remarks, Examples, Notations, and Terminologies  will be numbered
%%% in a single sequence, and the numbering will be within each
%%% section.  Displayed equations will be numbered in the same
%%% sequence. 
%%% 
%%% 
%%% Theorems, Propositions, Lemmas, and Corollaries will have the most
%%% formal typesetting.
%%% 
%%% Definitions will have the next level of formality.
%%% 
%%% Remarks, Examples, Notations, and Terminologies will be the least
%%% formal.
%%% 
%%% Theorem:
%%%% \begin{thm}
%%% 
%%% \end{thm}
%%% 
%%% Corollary:
%%% \begin{cor}
%%% 
%%% \end{cor}
%%% 
%%% Lemma:
%%% \begin{lem}
%%% 
%%% \end{lem}
%%% 
%%% Proposition:
%%% \begin{prop}
%%% 
%%% \end{prop}
%%% 
%%% Definition:
%%% \begin{defn}
%%% 
%%% \end{defn}
%%% 
%%% Remark:
%%% \begin{rem}
%%% 
%%% \end{rem}
%%% 
%%% Example:
%%% \begin{ex}
%%% 
%%% \end{ex}
%%% 
%%% Notation:
%%% \begin{notation}
%%% 
%%% \end{notation}
%%% 
%%% Terminology:
%%% \begin{terminology}
%%% 
%%% \end{terminology}
%%% 
%%%       Theorem environments% }~}~}'

% The following causes equations to be numbered within sections
\numberwithin{equation}{section}


\theoremstyle{plain} %% This is the default, anyway

\newtheorem{cor}[equation]{Corollary}
\newtheorem{lem}[equation]{Lemma}
\newtheorem{prop}[equation]{Proposition}

\theoremstyle{definition}
\newtheorem{defn}[equation]{Definition}

\theoremstyle{remark}
\newtheorem{rem}[equation]{Remark}
\newtheorem{ex}[equation]{Example}
\newtheorem{notation}[equation]{Notation}
\newtheorem{terminology}[equation]{Terminology}

% }~}~}'
%%%-------------------------------------------------------------------
\begin{document}

%%% In the title, use a double backslash "\\" to show a linebreak:
%%% Use one of the following two forms:
%%% \title{Text of the title}
%%% or
%%% \title[Short form for the running head]{Text of the title}
\title{Pre-calculus}


%%% If there are multiple authors, they're described one at a time:
%%% First author: \author{} \address{} \curraddr{} \email{} \thanks{}
%%% Second author: \author{} \address{} \curraddr{} \email{} \thanks{}
%%% Third author: \author{} \address{} \curraddr{} \email{} \thanks{}
\author{Prison Math Project}
\date{}


\maketitle

\textbf{\textit{A Note to the Reader}}% '{~{~{

Hello! What follows is a sequence of modules \footnote{A \textit{module} is a self-contained collection of materials that are organized so that a student can learn a subject without the aid of an instructor.
By self-contained, we mean that the student do  not need supplementary resources to aid in their studies. However, if you have an extra pre-calculus book on hand, you certainly are not discouraged from using it 
for extra practice.} intended to support you in learning Pre-Calculus. Pre-Calculus is typically a review and supplimentary course before the Calculus courses. It is a large review of Algebra, including the study
of functions, graphs and their properties. Especially polynomials and trigonometric funcitons which feature a prominent role in Calculus. (If these words are unfamiliar don't be scared we'll cover them in time in more
detail)

The Prison Math Project (PMP) realizes that you may be practicing mathematics in an environment that is highly restrictive. In response, this text is designed to be used independently, without an instructor, 
and also \textit{does not require a calculator}. Once you complete this module on Fundamentals of Real Numbers, you are free to move onto the next module topic: Exponents and Radicals. Exponents and Radicals are
ways in which we express the ideas of repeated multiplication and the un-doing of that such as the square-root. But before we jump too far ahead, let us work and (re)learn the properties of Real Numbers.% }~}~}'


\textbf{\textit{How to learn with this set of modules.}}% '{~{~{

 Arguably, the best way to learn mathematics is to solve problems. Throughout the text you will find loads of problems. There are three types:
 \textbf{\textit{Progress Check Problems}}, problems found at the end of a section which we will plainly call \textbf{\textit{Problems}},
 and then more challenging problems found at the end of sections which we call \textbf{\textit{Challenge Problems}}.
 Our recommendation is that you make valiant efforts to solve all the problems you run across. 
 
 - \textbf{\textit{Progress Check Problems}} are spread throughout sections. They are meant help guide your thinking as you read through the text.
 They also serve as a gauge to help you determine how well you are understanding what you are reading. Reading a math text is a lot different than 
 reading ordinary books. Problems need to be solved throughout. At the end of each progress check problem, you will find its solution.
 It is strongly encouraged that you make efforts to solve each problems before looking at its corresponding solution.

-  \textbf{\textit{Problems}} at the end of each section solidify your understanding of all topics that were discussed in the section.
It is recommended that you solve (or at least understand each solution) each problem before moving on to the next section. Solutions to \textbf{\textit{Problems}} 
are found at the end of the module. Try your best not to reference the solution until you have given honest efforts in solving the problem! This cannot be stressed enough.
Referencing the solution before you have given yourself time to think through a problem will take away your opportunity to truly learn the material.
If you get stuck on a problem, go back and re-read the section or work through examples carefully. Getting stuck is part of the learning process!
All mathematicians get stuck at some point in their career. We also recommend that if you do get stuck, write down the problem you are having difficulties with and write a letter to PMP.
We will respond as quickly as we can and can even match you up with a Mentor so that you have someone to work through the problems with. We are here to help!
The beautiful part of doing mathematics is that you never have to do it completely alone.

- \textbf{\textit{Challenge Problems}} are harder than \textbf{\textit{Problems}}, generally. These problems will require more time and thought.
They are not required to be solved in their entirety before moving on to the next module; however, it is strongly recommended to understand their solutions.
You will develop helpful problem solving techniques and tricks if you can understand the challenge problems' solutions. Besides,
if you can find success in solving the challenge problems, the normal problems may come easier.

- \textbf{\textit{Reflection Prompts}} are not problems in the traditional sense, but they are embedded throughout modules to give you space to reflect 
on your experience in interacting with the content. In this space you can make suggestions to PMP about how to make the module work better for you.

% }~}~}'

\vspace{0.3cm}

That's it! You're ready to begin your journey of learning pre-calculus and join a long history of mathematicians and scientists who studied this beautiful subject. % '{~{~{
Pre-Calculus by it's nature is a review and is a modern interpretation of what is needed to move on to Calculus. It's a collection of ideas from Algebra and Trigonometry among other areas of Math. One of the most
interesting parts of Pre-Calculus is the study of the solutions to polynomial equations along with the development of \emph{Coordinate Geometry} as a tool for working with algebra and geometry simultaneously!

This allows tools from one to be used in the other. Quite a neat trick!% }~}~}'

\newpage


\textit{From the author, R.S.:}% '{~{~{

\textit{\small Math has always been the safe-haven of my mind. It has always given me something to do when I'm bored out of my mind, wherever that may be. It will be hard at times and that's alright, that's normal, and it's a part of the process. I hope you cand find the beauty I did. Good Luck} % }~}~}'




\newpage

%%% To include a table of contents, uncomment the following line:

\tableofcontents

%%%-------------------------------------------------------------------
%%%-------------------------------------------------------------------
%%% Start the body of the paper here!  E.G., maybe use:




\newpage

\section{\textbf{Real Numbers}}
This Chapter of the module is meant to help make sure the reader is comfortable with the idea of a \emph{Real Number} and all of the other Numbers that came before them and are a part of them.
It's also a refresher on some of the terminology that might differ from everyday usage and understanding. Mathematicians are very particular with their language and so will we in this text.

\subsection{What Numbers Do We Know?}%
\label{sub:What numbers do we know?}
\\~\\

Throughout your current education, however you may have gotten it, you likely have learned about numbers. Specifically we can start with the idea of a counting number for instance:
\[
	1,2,3 \dots
\]
Mathematicians generally call these numbers \emph{Natural Numbers} because they are the `Most natural' numbers to first create. (Side-note, often-times and often not Mathematicians will include the 
number $0$ in the natural numbers. Though this depends on the area of math, and the mathematician!).

After the Natural numbers comes the extension to the \emph{Integers}. Which includes every counting number,
and the negative version of it. That is to say
\[
	\cdots -3, -2, -1 ,0 ,1 ,2 ,3 \dots
\]
This may not seem like such an important jump, but negative numbers were often avoided at all costs let alone able to exist on their own.

Once you have all of the integers what can you do with them? You likely know that you can add and subtract them just fine. And even multiply, but you cannot always divide, at least evenly that is. That's until we
modify our notion of dividing evenly.

We basically say in defiance, `Oh, you say I can't divide 3 by 2 evenly. Well watch me\dots it's $\frac{3}{2} $'. We do this for all divisions that could make sense \footnote{
That is to say, since dividing by 0 doesn't make sense, dividing any whole number by any other whole number that's not 0}
And there are many ways to represent this number,
such as it's decimal form $1.5$ or by writing $1\frac{1}{2} $ to mean that it is $1$ whole with an additional $\frac{1}{2} $. Although this mixed fraction is nicer to get an idea of size, it isn't very easy
to do arithmetic with, especially multiplication and division. For that reason we are going to stick with the from $\frac{3}{2} $ and keep this true for all other fractions that we might want to change
to mixed form.

Back on topic, what are we too call all of these divisions of integers? And does it fix the issue of division not always working? Firstly Mathematicians have settled on the name \emph{Rationals}
Because each of these numbers is Rational, meaning that it is the ratio of two whole numbers. Secondly: Yes! We'll see shorly the rules for dealing with fractions and these rational numbers. But as we will see
if we divide two rational numbers (so long as what we are dividing by is not zero) then we end up with a rational number again.

To be careful we are going to make a definition of what we mean by rational numbers:

\begin{defn}{\textbf{Rational Numbers}}
	A number, call it $r$ is said to be a \emph{Rational Number} if it can be expressed as the ratio of two integers:
	\[
	r = \frac{m}{n} 	
	\]
	Where $m,n$ are integers and $n\neq 0$
\end{defn}


%======================
%	TODO
% Author: rmathguy
% 03-18-23 (M-D-Y)
% Finish Writing about the different number systems espeically all the fraction rules
% that we'll expect the reader to be comfortable with before they continue
% 
% Secondly, mention that all numbers are not rational, introduce the idea of irrational numbers
% probalby without proof of irrationality.

%======================
%	TODO
% Author: rmathguy
% 03-18-23 (M-D-Y)
% After That is done talk about exponents! and have a little fun :)
\end{document}
